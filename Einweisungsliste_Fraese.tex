%%%%%%%%%%%%%%%%%%%%%%%%%%%%%%%%%%%%%%%%%%%%%%%%
% COPYRIGHT: (C) 2012-2015 FAU FabLab and others
% CC-BY-SA 3.0
%%%%%%%%%%%%%%%%%%%%%%%%%%%%%%%%%%%%%%%%%%%%%%%%


\newcommand{\basedir}{./fablab-document}
\documentclass{\basedir/fablab-document}
\usepackage{fancybox} %ovale Boxen für Knöpfe
\usepackage{amssymb} % Symbole für Knöpfe
\usepackage{subfigure,caption}
\usepackage{eurosym}
\usepackage{tabularx} % Tabellen mit bestimmtem Breitenverhältnis der Spalten
\usepackage{wrapfig} % Textumlauf um Bilder
\renewcommand{\texteuro}{\euro}
\usepackage{ifthen}
\usepackage{xspace}
\def\tabularnewcol{&\xspace} % hässlicher Workaround von http://tex.stackexchange.com/questions/7590/how-to-programmatically-make-tabular-rows-using-whiledo


\usepackage{tabularx} % Tabelle mit teilweise gleich großen Spalten
\title{Einweisungsliste Fräse}
\fancyfoot[C]{github.com/fau-fablab/fraese-einweisung}
\fancyfoot[L]{Einweisungsliste Nr. \underline{\hspace{3em}}}

\begin{document}
%\maketitle

Ich bestätige mit meiner Unterschrift verbindlich, dass

\begin{itemize}
\item ich nur bestimmte Tätigkeiten selbstständig machen darf und andere nur unter direkter Aufsicht.\\
Stufe 1: Werkstückwechsel/Putzen, 2: Handbetrieb und Vorbereitung, 3: Alles, auch Start/Resync. \\
\textbf{Dies ist am Anfang der Einweisung genauer beschrieben.}
\item ich die Regeln und Hinweise (Kapitel 1 der Einweisung), bzw. ab Stufe 2 die komplette Einweisung, gelesen und verstanden habe.
\item ab Stufe 2: Ich die Zerspanungs-Mailingliste abonniert habe, um wichtige Neuerungen zu erfahren.
\item die Möglichkeit für Fragen und Diskussion gegeben war
\item die Möglichkeit, eine Kopie der Anleitung zu erhalten, gegeben war
\end{itemize}

Die Einweisung verfällt nach 1 Jahr, wenn sie nicht vorher erneuert wurde.

\newcommand{\quer}[1]{\rotatebox{90}{\textbf{#1}\hspace{1em}}}
% einfach kopiert von Einweisungsliste Lasercutter
\newcounter{i}
\setcounter{i}{1}
\newcommand{\leerezeile}{\hspace{2em} \tabularnewcol \hspace{3em} \tabularnewcol \vbox{\vspace{3em}} \tabularnewcol  \tabularnewcol  \tabularnewcol  \tabularnewline \hline}
%
\begin{tabularx}{\textwidth}{|l|l|X|l|X|X|X|X|}
  \hline
  \textbf{Nr.} & \textbf{Datum} & \textbf{Name, Unterschrift} & \textbf{Stufe 1/2/3} & \textbf{Werkstück} \mbox{(\enquote{Lehrlingsstück})}& \textbf{Einweisender}, \mbox{\textbf{Unterschrift}} \\ \hline
  \whiledo{\value{i}<9}%
  {%
    \stepcounter{i} \leerezeile
  }%
  \leerezeile % doofer Workaround, eigentlich sollte das auch in der Forschleife gehen! Ohne dies wird die Spaltenbegrenzung von Spalte 1 zu weit gezeichnet.
\end{tabularx}

\end{document}